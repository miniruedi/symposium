\begin{conf-abstract}
{Results of gas measurements during a pump test in Cornwall}
{Bettina Strauch and Martin Zimmer}
{GFZ Potsdam, Germany}
{Within the EU funded project “CRM geothermal”, that aims to establish an overview of the potential of geothermal fluids for raw material extraction, a pump test was conducted at a drill site in Cornwall, UK to assess the composition of the produced geothermal liquid. 
The borehole was drilled in 2019 to 1100m depth by Cornish Lithium Company in United Downs, Cornwall, UK for lithium exploration. The well crosses two permeable structures at approximately 600 m and 1011 m where low-salinity geothermal waters are hosted in natural fractures of granite and a metamorphic aureole. The water has an elevated lithium concentration due to the dissolution of lithium-enriched minerals in the granite. 
During a test campaign in summer 2023, a production test was conducted from 19/06/2023 to 22/06/2023. Beside the dissolved ions, also the gas composition was monitored during pumping operation. The focus was here on Helium which is of economic importance and, in view of emerging digital applications, assumed to become critical (high demand, low availability). 
An elevated helium content in the produced gas was expected, as the host granite contains large amounts of radionuclides, such as Uranium and Thorium, that results in Helium production upon decay. 
The sampling campaign was accompanied by an online gas monitoring of the headspace gas using the MiniRuedi. Furthermore, the GMIMS was used for gas-water separation.
In addition, experiments addressing the option of online-helium extraction using an alternative membrane-based gas extraction method, were performed. 
A conventional gas sampling for lab-based analyses was completed as well. 
The data showed up to 1 vol.\% Helium and a good agreement between different extraction and measurement techniques. can be attested. The data evaluation is still ongoing and preliminary results will be presented.}
\end{conf-abstract}
