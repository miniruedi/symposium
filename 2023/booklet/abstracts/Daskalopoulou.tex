\begin{conf-abstract}
{Combining and comparing systematic and continuous gas monitoring techniques at the Hartoušov mofette, Czech Republic}
{Kyriaki Daskalopoulou$^{1,2}$, Martin Zimmer$^{2}$, Heiko Woith$^{2}$, Samuel Niedermann$^{2}$, Andrea Vieth-Hillebrand$^{2}$, Walter D´Alessandro$^{3}$, Fausto Grassa$^{4}$, Josef Vlček$^{4}$}
{1 University of Potsdam Germany, 2 GFZ Potsdam Germany, 3 INGV-sezione di Palermo Italy, 4 Charles University Prague Czechia}
{The Cheb Basin (Czech Republic) is an intraplate region without active volcanism that is characterised by emanations of magma-derived gases and the occurrence of mid-crustal earthquake swarms. Associated intense mantle degassing occurs at the Hartoušov mofette field, a representative site for the Cheb Basin. Gases ascending from two boreholes (F1, $\sim$28m depth and F2, $\sim$108m depth) and two bubbling ponds (surface expressions) have been sampled systematically since 2019. Samples were analysed for their chemical (CO2, N2, O2, Ar, He, CH4, and H2) and isotopic contents (noble gases and CO2). CO2 is the dominant gas species (concentrations $>$99.3\%), while the remaining gases are present in minor amounts. The He isotopic composition is typical for the subcontinental lithospheric mantle ($\sim$5.4 to 5.9RA), and the $\delta$13CCO2 data reflect mantle-like CO2 (-2.4 to -1.3\permil vs. V-PDB). Variations in the chemical composition are mainly mirrored in the minor components and, similar to the $\delta$13CCO2 shifts, are often related to changes in water temperature, water/gas ratio, solubility differences, and the impact of microbial activity. To better document these changes, a multidisciplinary observatory was built at the Hartoušov mofette. There, gas composition from F2 is continuously analysed in-situ through a quadrupole mass spectrometer. Radon concentration and fluid parameters (water temperature, water level/pressure, gas pressure) are continuously measured at 3 different depth levels. In addition to the fluid monitoring equipment, a weather station records meteorological standard parameters, and a seismometer documents earthquake activity. This work will present data from the systematic and continuous gas monitoring and will discuss the problematics of continuous monitoring that may occur in systems characterised by extreme conditions.}
\end{conf-abstract}
