\begin{conf-abstract}
{The search for microbial activity}
{Anneleen Vanleeuw}
{SCK CEN, Belgium}
{Safe geological disposal of radioactive waste necessitates understanding the geochemical dynamics and microbial activities that show an interplay with these dynamics. A seven-year in situ monitoring study on Boom Clay pore water revealed significant fluctuations in dissolved CO2 concentrations, in combination with a progressive increase in dissolved methane concentrations over time. The latter phenomenon was observed in correlation with increasing pH levels. The empirical finding of the rising methane concentrations complicates the further development of geochemical models for Boom Clay pore water, thereby underscoring the need for a better understanding of contributing factors that influence pore water composition, including microbial activities such as methanogenesis. 
The bentonite buffer of the multi-barrier concept is expected to play an important role in precluding microbial activity for the purpose of limiting any negative microbial impact on corrosion rates. Not only the underlying mechanism constraining microbial activity in high density bentonite but also to what extend microbial activity in such high density bentonite can affect corrosion remains unclear. Therefore, an oedometer based experimental setup developed at SCK CEN was used to study microbial corrosion of carbon steel at a bentonite dry density of 1.6 g/cm³. Bentonite powder (MX-80) was added to the oedometers and 4 carbon steel coupons were placed in the MX-80 such that all coupons were completely covered with bentonite. Afterwards, the oedometers were closed and percolation was initiated with sterile synthetic Opalinus Clay water with or without 1.5 bar of a H2:CO2 (80:20 v/v) mixture. After full saturation, water and gasses are being collected and monitored for their microbial and chemical composition 
As the presence and concentration of gas is a very important indicator for several microbial processes, accurate analysis of the CO2, CH4 and H2 concentration in the gas phase is essential. Given its low detection limits, small gas volume consumption and the possibility for in-line analysis, Mini-Ruedi is considered to be an appropriate solution for the gas analysis in microbial experiments.}
\end{conf-abstract}
