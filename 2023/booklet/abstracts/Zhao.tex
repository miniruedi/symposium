\begin{conf-abstract}
{Groundwater recharge and paleoclimate deduced from noble gas with miniRUEDI measurement and other methods in arid Qaidam basin, China}
{Dan Zhao and Guangcai Wang}
{China University of Geosciences, Beijing}
{The Qaidam Basin (QB) is an extremely arid endorheic basin with high altitudes (2600 to 3000 m excluding mountainous areas) in northwest China. Huge aquifers, dozens of alluvial-lacustrine plains, as well as salt lakes and playas formed due to its hydrogeological and climatic circumstances. It’s very important and interesting to discover the groundwater recharge and its connection with local paleoclimate for sustainable utilization of water resources.

Here we present a case study on the groundwater recharge and paleoclimate with multiple methods in the Golmud water system, south part of QB. Field and lab work were conducted in 2017-2018. (i) Noble gases were performed with both in-situ miniRUEDI measurement and lab analysis for noble gas temperatures (NGTs). (ii) 3H, 3H-3He and 14C dating methods were used for the recharge period identification. (iii) Hydrochemistry, stable isotopes (2H and 18O), radioactive 222Rn, and in-situ physical-chemical parameters (temperature, pH, EC, etc.) were also employed for the groundwater recharge features and groundwater-surface water interaction characterization.

The preliminary results showed that (i) miniRUEDI provided a quick and relatively accurate measurement of gases (O2, N2, CO2, 40Ar, 4He and 84Kr) in air and water. Difference with lab analysis was found in some samples, but the results were comparable. (ii) MiniRUEDI measurement helps a lot in conveniently field-distinguishing ancient groundwater with high 4He and low O2 contents. (iii) Local atmospheric air showed stable percentage of different gases, but with lower air pressure (P<0.75 atm) and gases concentrations due to the high altitude (H from 2700 to 3700 m). Meanwhile, the air P-H relationship basically matches the equation P=e(-H/8300). (iv) Modern and late Pleistocene groundwater was identified. And NGTs result presents a gradually warming-up process of about 3℃ in 17.2-14.8 ka BP (corrected 14C age) after the Last Glacial Maximum in QB.

Acknowledgments: this research was supported by the National Natural Science Foundation of China (No. 41672243). The authors greatly appreciate the help from Dr. Edith Horstmann with noble gas analysis. Wan-Jun Jiang, Liang Guo, Fu Liao, Nuan Yang are also kindly acknowledged for their fieldwork assistance.}
\end{conf-abstract}
