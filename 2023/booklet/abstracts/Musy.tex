\begin{conf-abstract}
{Online monitoring to shed light on mixing dynamics in the aquifers of Mt. Fuji (Japan)}
{Stephanie	Musy}
{University of Basel, Switzerland}
{
Mt. Fuji is the iconic centerpiece of a large, tectonically active volcanic watershed (100 km2) which plays a vital role in supplying safe drinking water to millions of people through groundwater and numerous freshwater springs. Situated at the top of the sole known continental triple-trench junction, the Fuji watershed experiences significant tectonic instability and pictures complex geology. Recently, the conventional understanding of Mt. Fuji catchment being a conceptually simple, laminar groundwater flow system with three isolated aquifers was challenged: the combined use of noble gases, vanadium, and microbial eDNA as measured in different waters around Fuji revealed the presence of substantial deep groundwater water upwelling along Japan’s tectonically most active fault system, the Fujikawa Kako Fault Zone (FKFZ) [1].

These findings call for even deeper investigations of the hydrogeology and the mixing dynamics within large-scale volcanic watersheds, which are typically characterized by complex geologies and extensive networks of fractures and faults. In our current study, we approach these questions by integrating existing and emerging methodologies, such as continuous monitoring of dissolved gases (GE-MIMS; e.g. [2]) and microbes [3], combined with discrete samplings of eDNA, trace elements, and environmental isotopes. The continuous monitoring is installed in a 100-m-deep pumping well hitting directly the FKFZ, where other tracers revealed a mixing between deep He-rich groundwater with freshly infiltrated water. The results are used to assess the response of the system to seismic activity and hydraulic forcings in the area. 

References:\\
1. Schilling et al., 2023, Nat. Water, 1:60-73. DOI: 10.1038/s44221-022-00001-4\\
2. Giroud et al., 2023, Front. Water, 4:1032094. DOI: 10.3389/frwa.2022.1032094\\
3. Besmer et al., 2016, Sci. Rep., 6:38462. DOI: 10.1038/srep38462
}
\end{conf-abstract}
