\begin{conf-abstract}
{Joint use of $^3$H/$^3$He apparent age and on-site helium analysis to identify groundwater flow dynamics and transport of PCE in an urban area}
{Christian Moeck$^1$, Andrea Popp$^{1,2}$, Matthias S. Brennwald$^1$, Mario Schirmer$^{1,3,4}$, Rolf Kipfer$^{1,2}$}
{1 Swiss Federal Insitute of Aquatic Science and Technology (Eawag), 2 Swiss Federal Insitute of Science and Technology (ETH), 3 University of Neuchâtel Switzerland, 4 Université Laval, Canada}
{For our urban study site in Northern Switzerland, we used stable water isotopes, chlorinated solvents, dissolved gas concentrations, and $^3$H and tritiogenic $^3$He concentrations to assess water flow paths and mixing between artificially infiltrated surface water and groundwater. Especially, the recent developments of portable field-operated gas equilibrium membrane inlet mass spectrometer (GE-MIMS) systems provide a unique opportunity to measure dissolved gas concentrations, such as $^4$He with a high temporal resolution at relatively low costs. Although the GE-MIMS are not capable of providing apparent water ages, $^4$He accumulation rates are often obtained from $^3$H/$^3$He ages and it has been shown that non-atmospheric $^4$He concentrations determined in the laboratory (e.g., by static (noble gas) mass spectrometry) and by field-based (GE-MIMS) methods closely agree. This agreement allowed us to establishing an inter-relationship between $^3$H/$^3$He apparent water ages and the non-atmospheric $^4$He excess (e.g., calibrating the $^4$He excess in terms of residence time). We demonstrate that the 4He excess concentrations derived from the GE-MIMS system serve as an adequate proxy for the experimentally demanding laboratory-based analyses. We combined the obtained water ages with hydrochemical data, water isotopes ($\delta^{18}$O and $\delta^{2}$H), and PCE concentrations to understand water flow dynamics. Moreover, we explain the origin and spatial distribution of PCE contamination found at our study site with our multi-tracer approach.}
\end{conf-abstract}
