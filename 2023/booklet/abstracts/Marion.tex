\begin{conf-abstract}
{Using (noble) gases as tracers to assess hydrology and gas dynamics in trees}
{Capucine Marion}
{Eawag, Swiss Federal Institute of Aquatic Science and Technology}
{Plants are a major control of the water exchange between the geosphere and the atmosphere. Despite the fact that plants annually transpire within a factor of 5 as much water as rivers discharge to the ocean, fluid and gas transport and other (tracer) hydrological aspects in plants are barely known. As roots take up water and gases from the soil and transport the fluids upward in the xylem, the supersaturation of dissolved atmospheric (noble) gases in soil and groundwater (excess air) might be used as a natural tracer to study the dynamics of water and gases in trees. By modifying techniques to determine (noble) gases in porous media we are developing an experimental method for real-time, in-situ and in-vivo analysis of dissolved gases in tree sap. The technique allows to continuously track fluid dynamics from the soil, through trees and other plants, into the atmosphere.
Semipermeable membrane probes were installed in the soil and at different heights in the stem of a small fir tree to sample the dissolved gases in the soil and in the xylem sap. Each probe was connected to a portable mass spectrometer (miniRUEDI) to analyze He, Ar, Kr, N2, O2, CO2, CH4 over weeks. Even the current experimental set up is not yet optimal we observed modulations in CO2 (and O2) abundance in response to plant-physiological processes within the tree. We also carried out artificial tracer experiments by watering the tree with He or Ar labelled water enabling to monitor the water uptake and transport in the tree.
In our contribution we will discuss these experiments and the potential of the developed methodology as new analytical tool assess the mutual relation between fluid dynamics and physiological processes in plants (e.g., drought induced cavitation in the vascular system).}
\end{conf-abstract}
