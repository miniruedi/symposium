\begin{conf-abstract}
{Measuring diffusion of gas in partly saturated clay}
{Elke Jacops}
{SCK CEN, Belgium}
{Within a deep geological repository (DGR) for nuclear waste, generation of gas is unavoidable. The main, initial transport mechanism is diffusion as dissolved species. In order to calculate a correct balance between gas generation and gas dissipation by diffusion, accurate knowledge of gas diffusion coefficients is essential. Currently, a large database is available for diffusion coefficients of different gases and different clay-based materials – but all tested samples were fully saturated. As desaturation of engineered barriers and host rock can occur in a DGR, diffusion has to be studied also under unsaturated conditions. Therefore, SCK CEN developed a new set-up and methodology to study diffusion of gases in unsaturated clay-based materials. The concept uses the double through-diffusion technique, with 2 gases diffusing in counter directions. In order to allow only diffusive transport, gas pressure is equal at both sides. The diffusion coefficient is calculated from the concentration increase of each gas in its downstream compartment. As the gas volume is limited, the amount of gas, used for sample analysis should be as small as possible. The most suitable analyser to measure the gas composition in the set-up is Mini-Ruedi. Initial experiments were performed using He and CH4 at one side, and Xe and C2H6 at the other side. Because of peak interference between CH4 and C2H6 and the extreme prices for Xe, it was decided to switch to He and Ar for the rest of the experimental matrix. The m/z ratios of 4 and 40 were obtained from the scans and used as base peaks for the measurement of helium and argon respectively. The standards used were of the composition 0.1\% (or 1000 ppm) He/Ar in 99.9\% Ar/He. Since the peak intensity from the argon peak is quite small relative to the helium peak in a standard gas of 99.9\% Helium and 0.1\% Argon and no other detectable peak in the spectrum was identified from the scans, the M detector was used for the tuning of the m/z scale. The base peaks have been used for all quantification and tuning. This method has proven to be quite efficient as the errors estimated from the sample gases are often under the 5\% margin from the analyses done so far.}
\end{conf-abstract}
