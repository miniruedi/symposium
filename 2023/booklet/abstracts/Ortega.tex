\begin{conf-abstract}
{Prospects for onsite gas applications in groundwater-dependent wetlands}
{Lucia Ortega Ormaechea}
{International Atomic Energy Agency (IAEA), Vienna, Austria}
{The effects of climate changes are threatening wetlands and groundwater systems, both directly and indirectly, through rising temperatures, changes of rainfall intensity and frequency, extreme climatic events like drought, flooding, and the frequency of storms. Although the impacts of climate change to water resources are yet relatively unknown, the potential impacts on surface water, particularly projected regional climate patterns and trends (i.e., climate variability and change) have been studied in some detail (Green et al., 2011). Yet, the effect of climate change on groundwater resources is poorly understood, especially how the systems will respond to climate change coupled with anthropogenic activities such as land use changes (Earman \& Dettinger, 2011; Havril et al., 2018). This is because groundwater systems are complex, and a combination of processes affect groundwater recharge, discharge, and quality. Moreover, visualisation of these impacts on groundwater is difficult, driving assumptions that the resource is more resilient than it is. In contrast to groundwater systems, wetlands provide strong visualisation of how water resources are impacted by both natural and anthropogenic processes. Yet, little is known about how groundwater resources can be assessed through the lens of wetland systems. 

To protect and manage groundwater resources through wetlands, it is necessary to understand the sources and sinks of water in the wetland, groundwater-surface water interactions, and the pathways of water within a catchment area at a range of spatial and temporal scales. The IAEA launched in 2022 a new Coordinated Research Project to explore the use of onsite noble gases, among other tracers, to assess groundwater resources through the lens of wetland systems for efficient water resources management in Member States. The project is still under implementation but some preliminary results of onsite noble gases have provided promising insights on recharge rates and groundwater origin.}
\end{conf-abstract}
