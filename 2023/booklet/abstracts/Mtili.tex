\begin{conf-abstract}
{Geochemical Background of Volatiles in Kyejo Area (Rungwe Volcanic Province) and Geochemical Composition of Soil Gas, Macro Seep and Gas Vent: Implication to CO$_2$ Prospect}
{Karim Mtili}
{Department of Geosciences, School of Mines and Geosciences, University of Dar es Salaam, Tanzania}
{The study focuses on the geochemical investigation of volatile compounds, with a specific emphasis on carbon dioxide (CO2), in the Kyejo area, situated within the Rungwe Volcanic Province, a part of the East African Rift System (EARS). The region is recognized for hosting one of the most active hydrothermal systems, closely associated with Quaternary volcanic activity. This leads to the active and passive diffusion of gases in various geochemical compartments, including the soil matrix, cold and hot springs, and gas vents. These gases are predominantly of magmatic origin and encompass a range of gaseous species, including CO2, nitrogen (N2), argon (Ar), oxygen (O2), sulfur dioxide (SO2), hydrogen sulfide (H2S), hydrogen chloride (HCl), hydrogen fluoride (HF), and noble gases such as helium (He).

In this study, particular attention is given to CO2, the second most abundant gas emitted by magma. Due to its low solubility and high mobility, CO2 serves as an effective tracer for detecting concealed natural resources. Areas exhibiting anomalous CO2 concentrations are indicative of subsurface gas systems, structural features like faults, preferential pathways, and geological units. To address these objectives, we conducted an examination of the distribution of CO2 within the soil pore matrix around the study area, with the assumption that gases originating from the subsurface sources migrate to near-surface environments where they can be detected. Furthermore, we provide insights into the geochemical composition of macro seeps found in cold springs and gas vents within the study area.

The research approach comprises two main components: first, soil gas sampling to assess the distribution of bulk gas concentrations across the study area, and second, sampling of high-purity gas vents and macro-seeps (cold springs) to identify the sources of these emissions. Utilizing a portable quadrupole mass spectrometer (MiniRuedi), we conducted on-site bulk gas composition analysis of collected samples. This study contributes to a better understanding of the geological and geochemical characteristics of the Kyejo area, specifically in relation to CO2 emissions, and implications to the economic prospectivity of this gas and effects to climate change.
}
\end{conf-abstract}
