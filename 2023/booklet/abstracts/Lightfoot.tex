\begin{conf-abstract}
{Continuous Measurements of Radon and Other Dissolved Gas Species in Groundwater: A Crucial Step in Earthquake Precursor Research?}
{Alexandra K.\ Lightfoot}
{Swiss Seismological Service (SED) at ETH Zurich, Switzerland}
{The ArtEmis project represents a new approach within the discipline of earthquake precursor research, focused on elucidating the relationship between radon (Rn) concentration fluctuations in groundwater and seismic events. The initial phase of the project entails measuring Rn concentrations at selected study sites by employing already available techniques, while in parallel developing a new Rn sensor, with increased sensitivity and at low-cost. In order to measure Rn concentrations at a high spatial resolution, further development and deployment of over 100 sensors is needed. Additional observables, such as groundwater temperature and acidity levels and other dissolved gas species will also be analysed. 
In preparation and to validate data to be obtained from the newly developed sensors, Rn analysis will be performed in advance and later in parallel utilising the currently available standard for analysing continuous Rn gas concentrations in groundwater (i.e., the Rad8). One location for pilot testing such continuous Rn analysis will be at the Bedretto tunnel and laboratory in Ticino, Switzerland, where groundwater channels are connected to existing fault lines. The Bedretto tunnel is located specifically around 1.5 km below the Swiss Alps, extending 5 km in length between Bedretto (Ticino) and the Furkapass in Switzerland. Due to its well-documented geological, seismo-tectonic and geochemical properties, and the fact that experiments on induced seismicity are conducted, the Bedretto site is ideal for real time monitoring of Rn concentrations. 
In addition to Rn, other dissolved noble and reactive gas species will simultaneously be analysed with a portable mass spectrometer, which is preceded by a gas permeable membrane-inlet system. Such initiatives are being pursued, given recent observations between seismic events and corresponding changes in dissolved noble and reactive gas concentration ratios in groundwater analysed in the Valais, Switzerland (Giroud, S. et al., 2022, doi: 10.46427/gold2022.8935).}
\end{conf-abstract}
