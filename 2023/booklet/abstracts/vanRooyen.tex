\begin{conf-abstract}
{Long term monitoring of noble gas and water isotope tracers in a localised MAR scheme to assess recharge and regional groundwater mixing dynamics}
{Jared van Rooyen}
{Eawag, Swiss Federal Institute of Aquatic Science and Technology}
{Managed aquifer recharge (MAR) has become increasingly popular in Central Europe as a sustainable, clean, and efficient method for managing domestic water supply. In these schemes, river water is artificially infiltrated into shallow aquifers for storage and natural purification of domestic water supply, while the resulting groundwater mound can simultaneously be designed such that it suppresses inflow of regional groundwater from contaminated areas. MAR schemes are typically not managed based on automated optimization algorithms, especially in complex urban and geological settings. However, such automated managing procedures are critical to guarantee safe drinking water. With (seasonal) water scarcity predicted to increase in Central Europe, improving the efficiency of MAR schemes will contribute to achieving several of the UN SDGs and EU agendas. Physico-chemical and isotope data has been collected over the last 3-4 decades around Switzerland’s largest MAR scheme in Basel, Switzerland, where 100 km3/d of Rhine river water are infiltrated and 40 km3/d are extracted for drinking water. The other 60 km3/d are used to maintain the groundwater mound that keeps locally contaminated groundwater from industrial heritage sites out of the drinking water. The hydrochemical/isotope data from past and ongoing studies were consolidated to contextualize all the contributing water sources of the scheme before online noble gas and regular tritium monitoring commenced in the region. The historical and the new continuous tracer monitoring data is now used to inform new sampling protocols and create tracer enabled/assimilated groundwater-surface water flow models, vastly helping algorithm-supported MAR optimization.}
\end{conf-abstract}
